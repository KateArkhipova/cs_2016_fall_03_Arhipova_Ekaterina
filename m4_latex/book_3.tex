\documentclass[a4paper]{article}

%% Language and font encodings
\usepackage[english]{babel}
\usepackage[utf8x]{inputenc}
\usepackage[T1]{fontenc}

%% Sets page size and margin

%% Useful packages
\usepackage{amsmath}
\usepackage{graphicx}
\usepackage[colorinlistoftodos]{todonotes}
\usepackage[colorlinks=true, allcolors=blue]{hyperref}
\setcounter{section}{1}
\setcounter{subsection}{1}
 \setcounter{equation}{2}
 \setcounter{page}{5}
\numberwithin{equation}{section}
\usepackage{fancyhdr}

\title{Your Paper}
\author{You}
%% Sets page size and margins 
\usepackage[a4paper,top=1cm,bottom=1cm,left=4cm,right=3cm,marginparwidth=1cm]{geometry} 


\begin {document}
\pagestyle{empty}
  \rhead{\thepage}
\subsection{\textit{Newtonian mechanics: Free fall \hspace{6 cm} 5}}
{\large One robust alternative is the method of dimensional analysis. But this
tool requires that at least one quantity among $v$, $g$, and $h$ have dimensions.
Otherwise, every candidate impact speed, no matter how absurd, equates
dimensionless quantities and therefore has valid dimensions.
\\

Therefore, let’s restate the free-fall problem so that the quantities retain
  their dimensions:}
\\

{\normalsize A ball initially at rest falls from a height $h$ and hits the ground at speed $v$.
Find $v$ assuming a gravitational acceleration g and neglecting air resistance} 
\\

{\large The restatement is, first, shorter and crisper than the original phrasing:}
\\

{\normalsize A ball initially at rest falls from a height of $h$ feet and hits the ground at a
speed of $v$ feet per second. Find v assuming a gravitational acceleration of $g$
feet per second squared and neglecting air resistance.}
\\

{\large Second, the restatement is more general. It makes no assumption about
the system of units, so it is useful even if meters, cubits, or furlongs are
the unit of length. Most importantly, the restatement gives dimensions to
$h$, $g$, and $v$. Their dimensions will almost uniquely determine the impact
speed—without our needing to solve a differential equation.
\\

The dimensions of height $h$ are simply length or, for short, $L$. The dimensions           
of gravitational acceleration $g$ are length per time squared or $LT^{−2}$,
where T represents the dimension of time. A speed has dimensions of
$LT^{−1}$, so $v$ is a function of g and h with dimensions of $LT^{−1}$.}
\\

\colorbox{lightgray}{
\begin{minipage}{\textwidth}
 {\normalsize\textbf{Problem 1.4 Dimensions of familiar quantities}

In terms of the basic dimensions length $L$, mass $M$, and time $T$, what are the
dimensions of energy, power, and torque?}
\end{minipage}
}
\\

$\Delta$\textit {What combination of $g$ and $h$ has dimensions of speed?}
\\

The combination $\sqrt{gh}$ has dimensions of speed.
\\

\begin{equation}
(\underbrace{LT^{−2}}_{g}\times \underbrace{L}_{h})^{1/2}=\sqrt{L^2T^{-2}}=\underbrace{LT^{-1}}_{speed}
\end{equation}
\\

$\Delta$ \textit {Is $\sqrt{gh}$ the only combination of g and h with dimensions of speed?}
\\

{\large In order to decide whether $\sqrt{gh}$ is the only possibility, use constraint
propagation [43]. The strongest constraint is that the combination of $g$ and
$h$, being a speed, should have dimensions of inverse time $(T^{−1})$. Because
h contains no dimensions of time, it cannot help construct $T^{−1}$. Because}


\newpage
 \large\textbf{6} \hfill \textit{1 Dimensions} \\ 
\vspace{0pt} 

$g$ contains $T^{−2}$, the $T^{−1}$ must come from $\sqrt{g}$. The second constraint is
that the combination contain $L^{1}$. The $\sqrt{g}$ already contributes $L^{1/2}$, so the
missing $L^{1/2}$ must come from $\sqrt{h}$. The two constraints thereby determine
uniquely how $g$ and $h$ appear in the impact speed $v$.
\\

The exact expression for $v$ is, however, not unique. It could be $\sqrt{gh}$,
$\sqrt{2gh}$,
or, in general, $\sqrt{gh}$×dimensionless constant. The idiom of multiplication
by a dimensionless constant occurs frequently and deserves a compact
notation akin to the equals sign:
\\

\begin{equation}
\nu \sim \sqrt{gh} 
\end{equation}
\\

Including this ∼ notation, we have several species of equality:
\\

\begin{center}
$\alpha$ \textrm{equality except perhaps for a factor with dimensions,}
\end{center}
\begin{eqnarray}
\sim \textrm{equality except perhaps for a factor without dimensions,}
\end{eqnarray}
\begin{center}
$\approx$ \textrm{equality except perhaps for a factor close to 1.}
\end{center}
The exact impact speed is $\sqrt{2gh}$, so the dimensions result $\sqrt{gh}$ contains
the entire functional dependence! It lacks only the dimensionless factor
$\sqrt{2}$, and these factors are often unimportant. In this example, the height
might vary from a few centimeters (a flea hopping) to a few meters (a cat
jumping from a ledge). The factor-of-100 variation in height contributes
a factor-of-10 variation in impact speed. Similarly, the gravitational acceleration
might vary from 0.27 m $s^{-2}$ (on the asteroid Ceres) to 25 m $s^{-2}$ (on
Jupiter). The factor-of-100 variation in $g$ contributes another factor-of-10
variation in impact speed. Much variation in the impact speed, therefore,
comes not from the dimensionless factor $\sqrt{2}$  but rather from the symbolic
factors—which are computed exactly by dimensional analysis.
\\

Furthermore, not calculating the exact answer can be an advantage. Exact
answers have all factors and terms, permitting less important information,
such as the dimensionless factor $\sqrt{2}$, to obscure important information
such as $\sqrt{gh}$. As William James advised, ‘‘The art of being wise is the art
of knowing what to overlook’’
[19, Chapter 22].
\\

\colorbox{lightgray}{
\begin{minipage}{\textwidth}
 {\normalsize\textbf{Problem 1.5 Vertical throw}
 \\
You throw a ball directly upward with speed $v0$. Use dimensional analysis to
estimate how long the ball takes to return to your hand (neglecting air resistance).
Then find the exact time by solving the free-fall differential equation. What
dimensionless factor was missing from the dimensional-analysis result?}
\end{minipage}
}

\newpage
\subsection{\textit{Guessing integrals} \hspace{8.8 cm} 7}
\subsection*{1.3\quad Guessing integrals}
The analysis of free fall (Section 1.2) shows the value of not separating
dimensioned quantities from their units. However, what if the quantities
are dimensionless, such as the 5 and $x$ in the following Gaussian integral:
\\

\begin{equation}
\int\limits_{-\infty}^{\infty} e^{-5x^2}\,dx?
\end{equation}
\\
Alternatively, the dimensions might be unspecified—a common case in
mathematics because it is a universal language. For example, probability
theory uses the Gaussian integral
\\

\begin{equation}
\int\limits_{x1}^{x2} e^{-x^2/2\sigma^2}\,dx,
\end{equation}
\\
where $x$ could be height, detector error, or much else. Thermal physics
uses the similar integral
\\

\begin{equation}
\int e^{-1/2m\nu^2/kT}\,d\nu,
\end{equation}
\\
where $v$ is a molecular speed. Mathematics, as the common language,
studies their common form studies their $\int e^{-\alpha x^2}$ without specifying the dimensions of
$\alpha$ and $x$. The lack of specificity gives mathematics its power of abstraction,
but it makes using dimensional analysis difficult.
\\

$\Delta$\textit {How can dimensional analysis be applied without losing the benefits of mathematical
abstraction?}
\\

The answer is to find the quantities with unspecified dimensions and then
to assign them a consistent set of dimensions. To illustrate the approach,
let’s apply it to the general definite Gaussian integral
\\

\begin{equation}
\int\limits_{-\infty}^{\infty} e^{-\alpha x^2}\,dx.
\end{equation}
\\
Unlike its specific cousin with $\alpha$ = 5, which is the integral $\int\limits_{-\infty}^{\infty} e^{-5x^2}$ the general form does not specify the dimensions of $x$ or $\alpha$—and that
openness provides the freedom needed to use the method of dimensional
analysis
\\

The method requires that any equation be dimensionally valid. Thus,
in the following equation, the left and right sides must have identical
dimensions:

\newpage
 \large\textbf{8} \hfill \textit{1 Dimensions} \\ 
\vspace{0pt} 

\begin{equation}
\int\limits_{-\infty}^{\infty} e^{-\alpha x^2}\,dx=something
\end{equation}
\\

$\Delta$\textit {Is the right side a function of $x$? Is it a function of $\alpha$? Does it contain a constant
of integration?}
\\

The left side contains no symbolic quantities other than $x$ and $\alpha$. But
$x$ is the integration variable and the integral is over a definite range, so
$x$ disappears upon integration (and no constant of integration appears).
Therefore, the right side—the ’’something‘‘
—is a function only of $\alpha$. In
symbols,
\\

\begin{equation}
\int\limits_{-\infty}^{\infty} e^{-\alpha x^2}\,dx=f(\alpha)
\end{equation}

The function f might include dimensionless numbers such as 2/3 or $\sqrt{\pi}$,
but $\alpha$ is its only input with dimensions.
\\

For the equation to be dimensionally valid, the integral must have the
same dimensions as f($\alpha$), and the dimensions of f($\alpha$) depend on the
dimensions of $\alpha$. Accordingly, the dimensional-analysis procedure has
the following three steps:
\\

Step 1. Assign dimensions to $\alpha$ (Section 1.3.1).
\\

Step 2. Find the dimensions of the integral (Section 1.3.2).
\\

Step 3. Make an f($\alpha$) with those dimensions (Section 1.3.3).
\\

\subsubsection{Assigning dimensions to $\alpha$}

The parameter $\alpha$ appears in an exponent. An exponent specifies how
many times to multiply a quantity by itself. For example, here is $2^{n}$:
\\

\begin{equation}
2^{n}=\underbrace{2 \times 2 \times... \times 2}_{n  terms}
\end{equation}
\\

The notion of ’’how many times‘‘ is a pure number, so an exponent is
dimensionless.
\\

Hence the exponent −$\alpha x^{2}$ in the Gaussian integral is dimensionless. For
convenience, denote the dimensions of $\alpha$ by [$\alpha$] and of $x$ by [$x$]. Then
\\
\begin{equation}
[\alpha][x^{2}]=1,
\end{equation}

\end{document}
